
\documentclass[11pt]{amsart}

\usepackage{geometry}
\geometry{letterpaper}
\usepackage[parfill]{parskip}
\usepackage{graphicx}
\usepackage{amssymb}
\usepackage{amsthm}
\usepackage{epstopdf}
\DeclareGraphicsRule{.tif}{png}{.png}{`convert #1 `dirname #1`/`basename #1 .tif`.png}

% theorems, definitions, proofs, etc.
\newtheorem{theorem}{Theorem}[section]
\theoremstyle{definition}
\newtheorem{definition}[theorem]{Definition}
\newtheorem*{remark}{Remark}
\newtheorem{corollary}[theorem]{Corollary}

% number systems
\newcommand{\complexNumbers}{\mathbb{C}}
\newcommand{\reals}{\mathbb{R}}
\newcommand{\rationals}{\mathbb{Q}}
\newcommand{\integers}{\mathbb{Z}}
\newcommand{\naturals}{\mathbb{N}}
\newcommand{\primes}{\mathbb{P}}

% specific character styling
\let\oldemptyset\emptyset
\let\emptyset\varnothing

% ring commands
\newcommand{\ring}{(R, +, \, \cdot)}
\newcommand{\ringSecondOp}{(R, \, \cdot)}

\title{Study Guide: Abstract Algebra}
\author{Phil Mayer}

\begin{document}
\maketitle

\section{Groups}
\begin{definition}
	$*$ is a \textbf{binary operation} if and only if $*: S \times S \to S$ is one-to-one and onto.
\end{definition}
\begin{definition}
	$(G, *)$ is a \textbf{group} if and only if $*$ is associative, has an identity in the set $G$, each $g \in G$ has an inverse in $G$,
	and $G$ is closed under the operation.
\end{definition}
\begin{definition}
	A group $G$ is \textbf{abelian} (or commutative) if and only if $\forall a, b \in G$, $ab = ba$.
\end{definition}
Groups have the following properties:
\begin{enumerate}
	\item{Cancellation law: $\forall a, b, c \in G, ab = ac \implies b = c$.}
	\item{Solution uniqueness: ``linear" equations of the form $ax = b$ have a unique solution in $G$.}
	\item{Uniqueness of identity: $e \in G$ is the only valid identity.}
	\item{Uniqueness of inverses: each $a \in G$ has a unique inverse $a^{-1} \in G$.}
	\item{Inverse of a product: $\forall a, b \in G$, $(ab)^{-1} = b^{-1}a^{-1}$.}
\end{enumerate}

\section{Subgroups and Cyclic Groups}
\begin{definition}
	$H \subset G$ is a \textbf{subgroup} of $G$ if and only if $H$ is closed under $G$'s binary operation, is associative, its identity
	is in $H$, and for each $h \in H$, $h^{-1} \in H$. We denote $H$ as a subgroup of $G$ by $H \leq G$.
\end{definition}
\begin{definition}
	$G$ is said to be a \textbf{cyclic group} generated by $a \in G$ if and only if $G = \langle a \rangle = \{ a^n \; | \; n \in \integers \}$.
\end{definition}
\begin{theorem}
	Any subgroup of a cyclic group is cyclic. This is a general statement that is difficult to prove, so below we will show that
	any subgroup of $(\integers, +)$ is cyclic.
\end{theorem}
\begin{proof}
	Suppose $H \leq G$ on addition. \\
	Case: if $H = \{ 0 \}$, then 0 generates $H$, so $H$ is trivially cyclic. \\
	Case: if $H \neq \{ 0 \}$, then we want to show $\exists d \in H$ such that $\langle d \rangle = H$. \\
	Since $H \neq \{ 0 \}$, $H$ contains a least positive integer $d$ since all non-empty subsets of $\integers$ have a
	least element. \\
	By closure of $H$, $\langle d \rangle \subset H$. Now we need to show $H \subset \langle d \rangle$. \\
	So let $h \in H$. We want $h = cd$ for some $c \in \integers$. \\
	Divide $h$ by $d$: then we have a quotient $q$ and remainder $r$ such that $h = dq + r$ with $0 \leq r < d$. \\
	Observe $r \in H$ since $r = h - dq$ and $h, dq \in H$ by closure. \\
	But since $d$ is the smallest integer in $H$ and $r < d$, $r = 0$. \\
	So $h = dq \implies h \in \langle d \rangle$. \\
	Then $H \subset \langle d \rangle$, so $H = \langle d \rangle$. \\
	$\therefore \; H$ is cyclic.
\end{proof}
\begin{theorem}
	There is exactly one cyclic subgroup of $\integers_n$ for each divisor $d$ of $n$, generated by $\frac{n}{d}$.
\end{theorem}
\begin{theorem}
	Some element $l \in \integers$ is a generator for $\integers_n$ if and only if $\gcd(\{ l, n \}) = 1$.
\end{theorem}
\begin{theorem}
	Integers $l, k \in \integers$ generate the same subgroup of $\integers_n$ if and only if $\gcd(\{ l, n \}) = \gcd(\{ k, n \})$.
\end{theorem}
\begin{theorem}
	Let $p \in \primes$, the prime numbers. Then $\integers_p$ has $p - 1$ generators and two distinct subgroups: $\{ 0 \}$ and 
	$\integers_p$.
\end{theorem}
\begin{theorem}
	Let $G$ be a cyclic group generated by $a$. If the order of $G$ is infinite, then $G \cong (\integers, +)$. If $G$ is finite with 
	order $n$, then $G \cong (\integers_n, +)$.
\end{theorem}

\section{Permutation Groups}
\begin{definition}
	A \textbf{permutation} is a one-to-one, onto function that rearranges a set. Composition of functions is well-known to be
	associative, have an identity, have inverses, and be closed.
\end{definition}
\begin{theorem}
	Let $A \neq \emptyset$ and call $S_A$ the collection of all permutations on a set $A$. Then $S_A$ is a group under 
	function composition.
\end{theorem}
\begin{theorem}
	If $n \geq 2$, then the collection of all even permutations of $\{ 1, \dots, n \}$ forms a subgroup $A_n$ of the symmetric group
	$S_n$ of order $\frac{n!}{2}$.
\end{theorem}
\begin{proof}
	It can be shown that $A_n$ is closed, has identity $e = (1,2)(1,2)$, and all inverses have an even number of elements. \\
	To show $| A_n | = \frac{n!}{2}$, we need to show $| A_n | = | B_n |$ by constructing a one-to-one, onto function between them. \\
	So let $f: A_n \to B_n$ by $f(\sigma) = \sigma(1,2)$. Then show one-to-one and onto.
\end{proof}
\begin{theorem}
	No permutation in $S_n$ can be expressed as both a product of an even and an odd number of transpositions.
\end{theorem}

\section{Cosets}
\begin{definition}
	Let $a \in G$ and suppose $H \leq G$. Then the \textbf{left coset} of $H$ in $G$ is the set $aH = \{ ah \; | \; h \in H \}$.
\end{definition}
Cosets have the following key properties:
\begin{enumerate}
	\item{$| aH | = | bH |$ for all cosets $aH$, $bH$ of $H$. Can be proven by constructing a one-to-one and onto function between.}
	\item{$aH = bH$ or $aH \cap bH = \emptyset$.}
	\item{$H$ is always a trivial coset of itself.}
\end{enumerate}
\begin{theorem}[The Theorem of LaGrange]
	Suppose $G$ is a finite group and $H \leq G$. Then $| H |$ divides $| G |$.
\end{theorem}
\begin{proof}
	Let $G$ be a finite group and suppose $H$ is a subgroup of $G$. \\
	Now decompose $G$ into a union of its left cosets. Assume there are $r$. Then we have: 
	\[ G = \bigcup_{i = 1}^{r} a_i H \]
	Expanded out, $| G | = | a_1 H \cup a_2 H \cup \dots \cup a_r H |$. \\
	Now recall that for two general sets $A$ and $B$, $| A \cup B | = |A| + |B| - |A \cap B|$. \\
	But since $a_i H \cap a_j H = \emptyset \; \forall i \neq j$, $| G | = | a_1 H | + | a_2 H | + \dots + | a_r H |$. \\
	Then since all cosets have the same order, $| G | = r | H |$. \\
	$\therefore \;$ $| H |$ divides $| G |$.	
\end{proof}
\begin{definition}
	The \textbf{index} of $H$ in $G$, $[ G: H ]$, is the number of distinct costs of H in G.
\end{definition}
\begin{corollary}[The Theorem of LaGrange]
	$\frac{|G|}{|H|} = [G : H]$
\end{corollary}

\section{Homomorphisms and Isomorphisms}
\begin{definition}
	The function $\phi: G \to G'$ is a \textbf{homomorphism} from $G$ to $G'$ if and only if $\phi(ab) = \phi(a)\phi(b) \; \forall a, b \in G$.
\end{definition}
\begin{definition}
	$\phi: G \to G'$ is an \textbf{isomorphism} from $G$ to $G'$ if and only if $\phi$ is a one-to-one, onto homomorphism.
\end{definition}
Recall that $\phi$ is one-to-one if and only if $\forall x_1, x_2 \in G$ where $\phi(x_1) = \phi(x_2)$, we have $x_1 = x_2$.
$\phi$ is onto if and only if $\forall g' \in G'$, $\exists g \in G$ such that $g' = \phi(g)$.
\begin{theorem}
	Assume $f: G \to G'$ is a homomorphism. Then:
	\begin{enumerate}
		\item{$f(e) = e'$}
		\item{$f(a^{-1}) = f(a)^{-1}$}
		\item{If $H \leq G$, then $f(H) \leq G'$}
		\item{If $K \leq G'$, then $f^{-1}(K) \leq G$}
	\end{enumerate}
\end{theorem}
\begin{definition}
	Let $f: G \to G'$ be a homomorphism. Then the \textbf{kernel} of $f$, $\ker(f)$, is the set of elements of $G$ which are sent to 
	the identity in $G'$. So $\ker(f) = \{ a \in G \; |; f(a) = e' \}$.
\end{definition}
\begin{theorem}
	$\ker(f) \leq G$ and $\frac{|G|}{| \ker(f) |} = | \text{image of $G$ under $f$} |$
\end{theorem}

\section{Factor Groups}
\begin{definition}
	$H \leq G$ is \textbf{normal}, denoted $H \vartriangleleft G$, if and only if $aH = Ha \; \forall a \in G$, or equivalently, $a^{-1}h_1 a = h_2$ 
	for $h_1, h_2 \in H$.
\end{definition}
\begin{theorem}
	$H \vartriangleleft G$ if and only if $[ G : H ] = 2$.
\end{theorem}
\begin{theorem}
	Let $H \leq G$. Then the left coset multiplication $(aH)(bH) = (ab)H$ is well-defined if and only if $H \vartriangleleft G$. The cosets form
	a group under multiplication: $G/H$.
\end{theorem}
\begin{theorem}[The Fundamental Theorem of Homomorphisms]
	The theorem relates factor groups, normal subgroups, and kernels of homomorphisms in three parts:
	\begin{enumerate}
		\item{If $f: G \to G'$ is an onto homomorphism, then $\ker(f) \vartriangleleft G$ and $G / \ker(f)$ is a group. }
		\item{If $H \vartriangleleft G$ and $f: G \to G/H$ by $f(g) = gH$, then $f$ is a homomorphism. }
		\item{If $f: G \to G'$ is an onto homomorphism, then $G / \ker(f) \cong G'$. }
	\end{enumerate}
\end{theorem}

\end{document}
